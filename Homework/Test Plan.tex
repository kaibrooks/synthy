% ECE 411
% Almatouq, Brooks, Forsman, Roman
% Fall 2019

% Document settings ------------------------------------------------
\documentclass[a4paper,12pt]{article}

% Commands ------------------------------------------------
\newcommand{\authorname}{Almatouq, Brooks, Forsman, Roman}
\newcommand{\classnumber}{ECE 411 - Group 18}
\newcommand{\projectname}{Project Theremizer - Test Plan - v1.0, 11/25/2019}

\newcommand{\figOverlay}{\put(34,10){\color{black!50} \figWatermark}} % Figure overlay settings
\newcommand{\figWatermark}{\small \authorname \; \today} 		% Figure overlay text
\newcommand{\figHere}{\begin{overpic}[percent,scale=0.34]}	% Settings for all figures
\newcommand{\figHereB}{\begin{overpic}[percent,scale=0.5]}	% A different setting

% Packages ------------------------------------------------

\usepackage[USenglish]{babel} 	% American English
\usepackage{blindtext}			% Generate latin crap
\usepackage[yyyymmdd]{datetime} % Sets date format to ISO 8601 standard
\renewcommand{\dateseparator}{-}% Sets date format to ISO 8601 standard

\usepackage{graphicx}			% Image importing and display
\graphicspath{ {images/} }		% Path to image folder
\usepackage{xcolor}				% Allows normal color words
\usepackage{color, colortbl}

\usepackage{float}				% Adds 'H' for figure placement location
\usepackage{enumitem}			% Use for QandA environment
\usepackage{booktabs}			% Merging columns in tables
\usepackage{pdfpages}			% Add a PDF

\usepackage[nostamp]{draftwatermark}	% use [nostamp] when finished, [firstpage] otherwise
\SetWatermarkText{DRAFT}
\SetWatermarkColor{red!50}
\SetWatermarkScale{3}

\usepackage{overpic}				% Puts text over figures
\usepackage[american]{circuitikz}	% American-style circuit diagrams

\usepackage{amsmath}				% Multi-line equations
\usepackage{caption}				% Equation caption formatting
\usepackage{physics}				% Easier derivatives
\usepackage{gensymb}				% Enable \degree for degree symbol
\usepackage{siunitx}				% SI units


\usepackage{array}					% Used for centering tabular data
\newcolumntype{M}[1]{>{\centering\arraybackslash}p{#1}} % The actual centered column format

\usepackage{listings} %For code in appendix

\definecolor{mymauve}{rgb}{0.58,0,0.82}
\definecolor{mygreen}{rgb}{0,0.6,0}
\definecolor{mygray}{rgb}{0.5,0.5,0.5}
\definecolor{ltgray}{rgb}{0.937, 0.937, 0.956}	% Divide standard RGB values by 255 for some reason 

% PSU colors
\definecolor{PSUgreen}{RGB}{106,127,16}
\definecolor{PSUltgreen}{RGB}{168,180,0}
\definecolor{PSUblue}{RGB}{0,117,154}
\definecolor{PSUltblue}{RGB}{161,216,224}
\definecolor{PSUgray}{RGB}{71,67,52}
\definecolor{PSUbrown}{RGB}{96,53,29}
\definecolor{PSUsienna}{RGB}{163,63,31}
\definecolor{PSUred}{RGB}{210,73,42}
\definecolor{PSUorange}{RGB}{220,155,50}
\definecolor{PSUyellow}{RGB}{230,220,143}
\definecolor{PSUtan}{RGB}{232,221,162}
\definecolor{PSUpurple}{RGB}{101,3,96}


\newenvironment{QandA}
	{\begin{enumerate}[label=\arabic*.]\sl} % Use slanted ques	tion text and Arabic numerals
  {\end{enumerate}}
\newenvironment{answered}{\par\normalfont}{} % Paragraph break and use normal font

% fancy header / footer lines
\usepackage{fancyhdr}% http://ctan.org/pkg/fancyhdr
\pagestyle{fancy}% Change page style to fancy
\fancyhf{}% Clear header/footer
\fancyhead[L]{\textcolor{PSUgray}{\classnumber}}
\fancyhead[R]{\textcolor{PSUgray}{\projectname}}
\fancyfoot[L]{\textcolor{PSUgray}{\authorname}}
\fancyfoot[R]{\textcolor{PSUgray}{\thepage}}
\renewcommand{\headrulewidth}{0.4pt}% Default \headrulewidth is 0.4pt
\renewcommand{\footrulewidth}{0.4pt}% Default \footrulewidth is 0pt


% Title Page ------------------------------------------------
\begin{document}
\lstset { %Formatting for code in appendix
  language=Matlab,
  basicstyle=\footnotesize\ttfamily,
  numbers=left,
  stepnumber=1,
  showstringspaces=false,
  tabsize=1,
  breaklines=true,
  breakatwhitespace=false,
  stringstyle=\color{mymauve},
  keywordstyle=\color{blue},
  commentstyle=\color{mygreen}, 
}


% Table of contents ------------------------------------------------
% not today motherfucker
\newpage
\tableofcontents


% Begin paper ------------------------------------------------
\newpage
\pagenumbering{arabic}
	\section{References}
    This test plan refers to the following documents:\\
    
    1) PDS.pdf, v1.0, 11/4/2019: The product design requirements specification for Team 18's 
    Theremizer project.
    
    2) Functional Decomposition of Theremizer.pdf, v1.0, 11/14/2019: The functional decomposition of Team 18's Theremizer project.
    
    3) Rubric for Test Plan and Test Cases, source: Dr. Faust's capstone page, link: http://web.cecs.pdx.edu/~faustm/capstone/forms/TestPlanRubric.pdf, date accessed: 11/29/2019
    
    4) Lecture 8 of Engineering Practices: "Testing and Documentation", source: Dr. Faust's ECE 411 page, link: http://web.cecs.pdx.edu/~faustm/ece411/lectures/Lecture8TestingAndDocumentation.pdf, date accessed: 11/29/2019
	\section{Objectives}
	Successful testing of Theremizer entails validation of correct PCB manufacture and circuit build,
    functional characterization and test for all individual functional blocks as per the functional 
    decomposition, correct functionality of integrated blocks, and verifying overall system performance 
    to the specifications called for by design.
	\subsection{Acceptance Test: Theremizer}
	Individual acceptance tests will be put in place for having a functioning theremin output, faithful reproduction of both unmodulated audio input and theremized audio input, filter controls for cutoff frequency and resonance, and meeting the engineering requirements of at least 1V user controllable output amplitude, at most \SI{600}{\ohm} output impedance, and at least \SI{10}{\kilo\ohm} input impedance. A unit that successfully meets all of these specifications will constitute one working Theremizer.
	\subsection{Functional Test}
    To verify each functional block, the following functional blocks must meet these parameters:\\
    \begin{itemize}
        \item All regulated power domains available (9V,5V,+/-15V).
        \item Audio preamplifier offers user controllable gain from 0 input volume to at least 30dB input gain.
        \item Local oscillator offers at least 1KHz frequency change with antenna proximity and a resting oscillation frequency above 200KHz.
        \item RF oscillator is tuneable to same frequency as local oscillator.
        \item Envelope modulator successfully mixes RF oscillator and audio input to a modulation index of at least 0.2.
        \item Final mixer stage outputs baseband audio and offers theremin output of at least 1KHz frequency sweep.
        \item Microprocessor provides control voltage within expected limits to the low pass filter.
        \item Low pass filter has useful resonance and cutoff frequency control.
        \item Output amplifier offers user controllable output gain from 0 volume to at least 3dB above the filter output.
    \end{itemize}
    \subsection{Integration Test}
    Successful integration of all of the functional modules can be seen by the following tests:
    \begin{itemize}
        \item RF oscillator and audio input are successfully mixed in envelope modulator.
        \item Envelope modulated RF and local oscillator are successfully mixed in frequency mixer stage.
        \item Low pass filter responds to microprocessor control voltages.
    \end{itemize}
	\subsection{Performance Test}
	To characterize the performance of the Theremizer, the following parameters will be looked at:
	\begin{itemize}
	    \item Frequency range of theremin
	    \item Range of useful tonal distance from theremin
	    \item Maximum input audio only output amplitude
	    \item Maximum theremin output amplitude
	    \item Maximum theremized audio output amplitudes
	    \item Minimum distance from EM interference sources to maintain operation
	    \item Resonance frequency range for resonance control
	    \item Frequency response of Theremizer at full range cutoff settings
	    \item Tuning stability over both time and power cycles
	\end{itemize}
	
	\section{Resources}
	\subsection{Personnel}
	Thorough testing of Theremizer will require strong test and measurement skills and the ability to interpret those results, physical circuit debugging skills, and hardware programming/debugging skills. The necessary personnel will be the 4 group members, engineers 1-4, all of whom have senior student level ECE skillsets:
	
	\\Engineers 1 and 2: Shall focus on physical validation of the circuit and performance/functionality of the analog modules. They will need a strong background in circuit assembly, analog circuit behavior, and schematic reference.
	
	\\Engineers 3 and 4: Shall focus on the validation of the hardware programming and microcontroller behavior and debugging any problems in communications with the microcontroller. They will require a strong background in computer engineering, the fundamentals of USB communications, and the Arduino code system including C/C++ compiling/linking and the Arduino IDE. 
	
	\subsection{Test Equipment}
	We will need the following equipment and their respective connectors and tools:
	\begin{itemize}
	\item Oscilloscope
	\item Function generator
	\item Multimeter
	\item Spectrum analyzer
	\item DC power supply
	\end{itemize}
	\\The test equipment can be found in the different engineering labs in the PSU College of Engineering. 
	
	\subsection{Other Equipment}
	We will also need the original prototype for reference and comparison, programming equipment for the microcontroller, and the means to listen to the audio output. We will need the following:
	\begin{itemize}
	\item Original analog circuit PCB Prototype
	\item and USB programming cable
	\item Arduino IDE
	\item Atmel Studio
	\item Windows 10
	\item USB drivers
	\item Audio Amplifier with speakers
	\end{itemize}
	\\The audio amplifier with the speakers will be provided by Jeff Roman.
	\subsection*{}

\end{document}
